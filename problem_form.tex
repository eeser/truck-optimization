%% bare_conf.tex
%% V1.4b
%% 2015/08/26
%% by Michael Shell
%% See:
%% http://www.michaelshell.org/
%% for current contact information.
%%
%% This is a skeleton file demonstrating the use of IEEEtran.cls
%% (requires IEEEtran.cls version 1.8b or later) with an IEEE
%% conference paper.
%%
%% Support sites:
%% http://www.michaelshell.org/tex/ieeetran/
%% http://www.ctan.org/pkg/ieeetran
%% and
%% http://www.ieee.org/

%%*************************************************************************
%% Legal Notice:
%% This code is offered as-is without any warranty either expressed or
%% implied; without even the implied warranty of MERCHANTABILITY or
%% FITNESS FOR A PARTICULAR PURPOSE! 
%% User assumes all risk.
%% In no event shall the IEEE or any contributor to this code be liable for
%% any damages or losses, including, but not limited to, incidental,
%% consequential, or any other damages, resulting from the use or misuse
%% of any information contained here.
%%
%% All comments are the opinions of their respective authors and are not
%% necessarily endorsed by the IEEE.
%%
%% This work is distributed under the LaTeX Project Public License (LPPL)
%% ( http://www.latex-project.org/ ) version 1.3, and may be freely used,
%% distributed and modified. A copy of the LPPL, version 1.3, is included
%% in the base LaTeX documentation of all distributions of LaTeX released
%% 2003/12/01 or later.
%% Retain all contribution notices and credits.
%% ** Modified files should be clearly indicated as such, including  **
%% ** renaming them and changing author support contact information. **
%%*************************************************************************


% *** Authors should verify (and, if needed, correct) their LaTeX system  ***
% *** with the testflow diagnostic prior to trusting their LaTeX platform ***
% *** with production work. The IEEE's font choices and paper sizes can   ***
% *** trigger bugs that do not appear when using other class files.       ***                          ***
% The testflow support page is at:
% http://www.michaelshell.org/tex/testflow/



\documentclass[conference]{IEEEtran}
% Some Computer Society conferences also require the compsoc mode option,
% but others use the standard conference format.
%
% If IEEEtran.cls has not been installed into the LaTeX system files,
% manually specify the path to it like:
% \documentclass[conference]{../sty/IEEEtran}





% Some very useful LaTeX packages include:
% (uncomment the ones you want to load)


% *** MISC UTILITY PACKAGES ***
%
%\usepackage{ifpdf}
% Heiko Oberdiek's ifpdf.sty is very useful if you need conditional
% compilation based on whether the output is pdf or dvi.
% usage:
% \ifpdf
%   % pdf code
% \else
%   % dvi code
% \fi
% The latest version of ifpdf.sty can be obtained from:
% http://www.ctan.org/pkg/ifpdf
% Also, note that IEEEtran.cls V1.7 and later provides a builtin
% \ifCLASSINFOpdf conditional that works the same way.
% When switching from latex to pdflatex and vice-versa, the compiler may
% have to be run twice to clear warning/error messages.






% *** CITATION PACKAGES ***
%
%\usepackage{cite}
% cite.sty was written by Donald Arseneau
% V1.6 and later of IEEEtran pre-defines the format of the cite.sty package
% \cite{} output to follow that of the IEEE. Loading the cite package will
% result in citation numbers being automatically sorted and properly
% "compressed/ranged". e.g., [1], [9], [2], [7], [5], [6] without using
% cite.sty will become [1], [2], [5]--[7], [9] using cite.sty. cite.sty's
% \cite will automatically add leading space, if needed. Use cite.sty's
% noadjust option (cite.sty V3.8 and later) if you want to turn this off
% such as if a citation ever needs to be enclosed in parenthesis.
% cite.sty is already installed on most LaTeX systems. Be sure and use
% version 5.0 (2009-03-20) and later if using hyperref.sty.
% The latest version can be obtained at:
% http://www.ctan.org/pkg/cite
% The documentation is contained in the cite.sty file itself.






% *** GRAPHICS RELATED PACKAGES ***
%
\ifCLASSINFOpdf
  \usepackage[pdftex]{graphicx}
  % declare the path(s) where your graphic files are
  \graphicspath{{./Images/}}
  % and their extensions so you won't have to specify these with
  % every instance of \includegraphics
  \DeclareGraphicsExtensions{.png}
\else
  % or other class option (dvipsone, dvipdf, if not using dvips). graphicx
  % will default to the driver specified in the system graphics.cfg if no
  % driver is specified.
  \usepackage[dvips]{graphicx}
  % declare the path(s) where your graphic files are
  \graphicspath{{./Images/}}
  % and their extensions so you won't have to specify these with
  % every instance of \includegraphics
  % \DeclareGraphicsExtensions{.png}
\fi
% graphicx was written by David Carlisle and Sebastian Rahtz. It is
% required if you want graphics, photos, etc. graphicx.sty is already
% installed on most LaTeX systems. The latest version and documentation
% can be obtained at: 
% http://www.ctan.org/pkg/graphicx
% Another good source of documentation is "Using Imported Graphics in
% LaTeX2e" by Keith Reckdahl which can be found at:
% http://www.ctan.org/pkg/epslatex
%
% latex, and pdflatex in dvi mode, support graphics in encapsulated
% postscript (.eps) format. pdflatex in pdf mode supports graphics
% in .pdf, .jpeg, .png and .mps (metapost) formats. Users should ensure
% that all non-photo figures use a vector format (.eps, .pdf, .mps) and
% not a bitmapped formats (.jpeg, .png). The IEEE frowns on bitmapped formats
% which can result in "jaggedy"/blurry rendering of lines and letters as
% well as large increases in file sizes.
%
% You can find documentation about the pdfTeX application at:
% http://www.tug.org/applications/pdftex



\usepackage{dblfloatfix}


% *** MATH PACKAGES ***
%
\usepackage{amsmath}
% A popular package from the American Mathematical Society that provides
% many useful and powerful commands for dealing with mathematics.
%
% Note that the amsmath package sets \interdisplaylinepenalty to 10000
% thus preventing page breaks from occurring within multiline equations. Use:
%\interdisplaylinepenalty=2500
% after loading amsmath to restore such page breaks as IEEEtran.cls normally
% does. amsmath.sty is already installed on most LaTeX systems. The latest
% version and documentation can be obtained at:
% http://www.ctan.org/pkg/amsmath





% *** SPECIALIZED LIST PACKAGES ***
%
\usepackage{algorithm2e}
\SetKwProg{Fn}{Procedure}{}{}
\DontPrintSemicolon
% algorithmic.sty was written by Peter Williams and Rogerio Brito.
% This package provides an algorithmic environment fo describing algorithms.
% You can use the algorithmic environment in-text or within a figure
% environment to provide for a floating algorithm. Do NOT use the algorithm
% floating environment provided by algorithm.sty (by the same authors) or
% algorithm2e.sty (by Christophe Fiorio) as the IEEE does not use dedicated
% algorithm float types and packages that provide these will not provide
% correct IEEE style captions. The latest version and documentation of
% algorithmic.sty can be obtained at:
% http://www.ctan.org/pkg/algorithms
% Also of interest may be the (relatively newer and more customizable)
% algorithmicx.sty package by Szasz Janos:
% http://www.ctan.org/pkg/algorithmicx




% *** ALIGNMENT PACKAGES ***
%
%\usepackage{array}
% Frank Mittelbach's and David Carlisle's array.sty patches and improves
% the standard LaTeX2e array and tabular environments to provide better
% appearance and additional user controls. As the default LaTeX2e table
% generation code is lacking to the point of almost being broken with
% respect to the quality of the end results, all users are strongly
% advised to use an enhanced (at the very least that provided by array.sty)
% set of table tools. array.sty is already installed on most systems. The
% latest version and documentation can be obtained at:
% http://www.ctan.org/pkg/array


% IEEEtran contains the IEEEeqnarray family of commands that can be used to
% generate multiline equations as well as matrices, tables, etc., of high
% quality.




% *** SUBFIGURE PACKAGES ***
%\ifCLASSOPTIONcompsoc
%  \usepackage[caption=false,font=normalsize,labelfont=sf,textfont=sf]{subfig}
%\else
%  \usepackage[caption=false,font=footnotesize]{subfig}
%\fi
% subfig.sty, written by Steven Douglas Cochran, is the modern replacement
% for subfigure.sty, the latter of which is no longer maintained and is
% incompatible with some LaTeX packages including fixltx2e. However,
% subfig.sty requires and automatically loads Axel Sommerfeldt's caption.sty
% which will override IEEEtran.cls' handling of captions and this will result
% in non-IEEE style figure/table captions. To prevent this problem, be sure
% and invoke subfig.sty's "caption=false" package option (available since
% subfig.sty version 1.3, 2005/06/28) as this is will preserve IEEEtran.cls
% handling of captions.
% Note that the Computer Society format requires a larger sans serif font
% than the serif footnote size font used in traditional IEEE formatting
% and thus the need to invoke different subfig.sty package options depending
% on whether compsoc mode has been enabled.
%
% The latest version and documentation of subfig.sty can be obtained at:
% http://www.ctan.org/pkg/subfig




% *** FLOAT PACKAGES ***
%
%\usepackage{fixltx2e}
% fixltx2e, the successor to the earlier fix2col.sty, was written by
% Frank Mittelbach and David Carlisle. This package corrects a few problems
% in the LaTeX2e kernel, the most notable of which is that in current
% LaTeX2e releases, the ordering of single and double column floats is not
% guaranteed to be preserved. Thus, an unpatched LaTeX2e can allow a
% single column figure to be placed prior to an earlier double column
% figure.
% Be aware that LaTeX2e kernels dated 2015 and later have fixltx2e.sty's
% corrections already built into the system in which case a warning will
% be issued if an attempt is made to load fixltx2e.sty as it is no longer
% needed.
% The latest version and documentation can be found at:
% http://www.ctan.org/pkg/fixltx2e


%\usepackage{stfloats}
% stfloats.sty was written by Sigitas Tolusis. This package gives LaTeX2e
% the ability to do double column floats at the bottom of the page as well
% as the top. (e.g., "\begin{figure*}[!b]" is not normally possible in
% LaTeX2e). It also provides a command:
%\fnbelowfloat
% to enable the placement of footnotes below bottom floats (the standard
% LaTeX2e kernel puts them above bottom floats). This is an invasive package
% which rewrites many portions of the LaTeX2e float routines. It may not work
% with other packages that modify the LaTeX2e float routines. The latest
% version and documentation can be obtained at:
% http://www.ctan.org/pkg/stfloats
% Do not use the stfloats baselinefloat ability as the IEEE does not allow
% \baselineskip to stretch. Authors submitting work to the IEEE should note
% that the IEEE rarely uses double column equations and that authors should try
% to avoid such use. Do not be tempted to use the cuted.sty or midfloat.sty
% packages (also by Sigitas Tolusis) as the IEEE does not format its papers in
% such ways.
% Do not attempt to use stfloats with fixltx2e as they are incompatible.
% Instead, use Morten Hogholm'a dblfloatfix which combines the features
% of both fixltx2e and stfloats:
%
% \usepackage{dblfloatfix}
% The latest version can be found at:
% http://www.ctan.org/pkg/dblfloatfix




% *** PDF, URL AND HYPERLINK PACKAGES ***
%
\usepackage{url}
% url.sty was written by Donald Arseneau. It provides better support for
% handling and breaking URLs. url.sty is already installed on most LaTeX
% systems. The latest version and documentation can be obtained at:
% http://www.ctan.org/pkg/url
% Basically, \url{my_url_here}.




% *** Do not adjust lengths that control margins, column widths, etc. ***
% *** Do not use packages that alter fonts (such as pslatex).         ***
% There should be no need to do such things with IEEEtran.cls V1.6 and later.
% (Unless specifically asked to do so by the journal or conference you plan
% to submit to, of course. )


% correct bad hyphenation here
\hyphenation{op-tical net-works semi-conduc-tor}

\def\footnoterule{\relax%
  \kern-5pt
  \hbox to \columnwidth{\hfill\vrule width 0.5\columnwidth height 0.4pt\hfill}
  \kern4.6pt}
\makeatother

\usepackage[usenames, dvipsnames]{color}

\begin{document}
\bstctlcite{IEEEexample:BSTcontrol}
%
% paper title
% Titles are generally capitalized except for words such as a, an, and, as,
% at, but, by, for, in, nor, of, on, or, the, to and up, which are usually
% not capitalized unless they are the first or last word of the title.
% Linebreaks \\ can be used within to get better formatting as desired.
% Do not put math or special symbols in the title.

\title{Working title...}
%\title{Tourist Trip Planning for\\Efficient and Balanced Smart Cities}

% author names and affiliations
% use a multiple column layout for up to three different
% affiliations

% \author{
% \IEEEauthorblockN{Petar Mrazovic\IEEEauthorrefmark{1}\IEEEauthorrefmark{2}, Josep L. Larriba-Pey\IEEEauthorrefmark{2}, Mihhail Matskin\IEEEauthorrefmark{1}}
%    \IEEEauthorblockA{\IEEEauthorrefmark{1}Department of Software and Computer Systems, Royal Institue of Technology\\
%    Stockholm, Sweden
%    \\\{mrazovic, misha\}@kth.se}
%    \IEEEauthorblockA{\IEEEauthorrefmark{2}Department of Computer Architecture, Polytechnic University of Catalonia\\
%    \\\{larri\}@ac.upc.edu}
%}

\DeclareRobustCommand*{\IEEEauthorrefmark}[1]{%
  \raisebox{0pt}[0pt][0pt]{\textsuperscript{\footnotesize\ensuremath{#1}}}}

\author{\IEEEauthorblockN{Elif Eser\IEEEauthorrefmark{1},
Petar Mrazovic\IEEEauthorrefmark{2}, Hakan Ferhatosmanoglu\IEEEauthorrefmark{3}, Josep L. Larriba-Pey\IEEEauthorrefmark{4},
Mihhail Matskin\IEEEauthorrefmark{5}}\vspace{3mm}
\IEEEauthorblockA{\IEEEauthorrefmark{1,3}Dept. of Computer Engineering, Bilkent University, Ankara, Turkey}
\IEEEauthorblockA{\IEEEauthorrefmark{2,5}Dept. of Software and Computer Systems, Royal Institute of Technology, Stockholm, Sweden}
\IEEEauthorblockA{\IEEEauthorrefmark{2,4}Dept. of Computer Architecture, Polytechnic University of Catalonia, Barcelona, Spain}
Email: \IEEEauthorrefmark{1}elif.eser@bilkent.edu.tr, \IEEEauthorrefmark{2}mrazovic@kth.se, \IEEEauthorrefmark{3}hakan@cs.bilkent.edu.tr, \IEEEauthorrefmark{4}larri@ac.upc.edu, \IEEEauthorrefmark{5}misha@kth.se}

% conference papers do not typically use \thanks and this command
% is locked out in conference mode. If really needed, such as for
% the acknowledgment of grants, issue a \IEEEoverridecommandlockouts
% after \documentclass

% for over three affiliations, or if they all won't fit within the width
% of the page, use this alternative format:
% 
%\author{\IEEEauthorblockN{Michael Shell\IEEEauthorrefmark{1},
%Homer Simpson\IEEEauthorrefmark{2},
%James Kirk\IEEEauthorrefmark{3}, 
%Montgomery Scott\IEEEauthorrefmark{3} and
%Eldon Tyrell\IEEEauthorrefmark{4}}
%\IEEEauthorblockA{\IEEEauthorrefmark{1}School of Electrical and Computer Engineering\\
%Georgia Institute of Technology,
%Atlanta, Georgia 30332--0250\\ Email: see http://www.michaelshell.org/contact.html}
%\IEEEauthorblockA{\IEEEauthorrefmark{2}Twentieth Century Fox, Springfield, USA\\
%Email: homer@thesimpsons.com}
%\IEEEauthorblockA{\IEEEauthorrefmark{3}Starfleet Academy, San Francisco, California 96678-2391\\
%Telephone: (800) 555--1212, Fax: (888) 555--1212}
%\IEEEauthorblockA{\IEEEauthorrefmark{4}Tyrell Inc., 123 Replicant Street, Los Angeles, California 90210--4321}}




% use for special paper notices
%\IEEEspecialpapernotice{(Invited Paper)}




% make the title area
\maketitle

% As a general rule, do not put math, special symbols or citations
% in the abstract
\begin{abstract}
\end{abstract}

\begin{IEEEkeywords}
\end{IEEEkeywords}

% For peer review papers, you can put extra information on the cover
% page as needed:
% \ifCLASSOPTIONpeerreview
% \begin{center} \bfseries EDICS Category: 3-BBND \end{center}
% \fi
%
% For peerreview papers, this IEEEtran command inserts a page break and
% creates the second title. It will be ignored for other modes.
\IEEEpeerreviewmaketitle

\section{Problem formulation}

\subsection{Formal problem description}
\label{subsec:formal}

The problem under study can be formally defined with the aid of an undirected graph $ G = (V, E) $, where $ V = \lbrace  v_1, ... , v_N \rbrace $ is the set of nodes representing delivery spots, and $ E = \lbrace e_{ij}: v_i,v_j \in V \rbrace $ %we may consider the directed version of it, as we have talked, to make it simple i remove i<j, e_ji also represents the same edge here
is the set of edges representing the routes between them. The travel time $ c_{ij} $ and the delivery duration $ d_i $ are associated with each edge $ e_{ij} \in E $ and node $ v_i \in V $, respectively. Additionally, each node $v_i \in V$ is assigned with the capacity $ l_i $ representing the number of the parking lots at the delivery spot $ v_i $.
%Each node $ v_i \in V $ is associated with the turnover model $ \mathcal{T}_i(t) $ which estimates the number of vehicles that occupy the delivery spot at the given time $ t $. we may use it in the next version.
Further, fleet of vehicles $ H = \{h_1, ... , h_M \} $, each assigned with the set of nodes to be visited $ D_k $, are available at the arbitrary depots $v_k^s \in V $ at the starting times $ t_k^s $. Finally, the goal of the proposed problem is to determine a set of $ M $ paths $ P = \{ p_k : h_k \in H \} $ that will minimize the total travel cost by avoiding congestions at the nodes, while visiting all of the planned nodes $ v_i \in D_k\; \forall h_k \in H $. The congestion at node $ v_i $ occurs when the number of corresponding visits at the certain time exceeds its capacity $ l_i $.

\begin{table}[b]
{\caption{Table of notations}}
\begin{tabular}{|p{1.5cm}|p{6cm}|}
\hline
\textbf{Notation} & \textbf{Description}\\
\hline
$ G(V,E) $ & Graph $G$ with the vertex set $V$ and edge set $E$ \\
\hline
$ v_i $ & A vertex representing a delivery spot \\
\hline
$ l_i $ & Capacity of a vertex $v_i$ \\
\hline
$ d_i $ & Delivery duration at $v_i$ \\
\hline
$ e_{ij} $ & An edge representing a route between $v_i$ and $v_j$\\
\hline
$ c_{ij} $ & Travel time needed to traverse $e_{ij}$ \\
\hline
$ h_k $ & Vehicle making delivery over $G$, $h_k \in H$  \\
\hline
$ D_k $ & Set of vertices to be visited by vehicle $h_k$ \\
\hline
$ v_k^s $ & Starting vertex (depot) of vehicle $h_k$, $v_s^k\in D_k$  \\
\hline
$ t_k^s $ &  Starting time of vehicle $h_k$ \\
\hline
$ p_k $ &  A solution path for vehicle $h_k$\\
\hline
$ x_{ij}^k $ & A decision variable to represent whether $e_{ij}$ is a part of $p_k$ \\
\hline
%s_i^k may be located here
$ t_{i}^k $ & Visiting time of $v_i$ in $p_k$ \\
\hline
$ y_{i,t}^k $ & A decision variable to represent whether $v_i$ is visited in $p_k$ at $t$ time \\
\hline
$ u_{i}^k $ & The position of $v_i$ in $p_k$ \\
\hline
\end{tabular}
\end{table}

\subsection{Mixed-integer linear programming model}
\label{subsec:milp}

We continue with the formalization by formulating our problem as a Mixed-integer Liner Programming Problem (MILP). The model is not intended to be used as an efficient solution to the proposed problem, but rather to better understand its complex structure.

We first define the following decision variables:

\begin{equation}
\label{eq:var_x}
x^k_{ij}=\begin{cases} 
      1 & e_{ij} \in p_k \\
      0 & otherwise 
   \end{cases}
\end{equation}

\begin{equation}
\label{eq:var_y}
y^k_{i,t}=\begin{cases} 
      1 & t^k_i \le t \wedge t^k_i + d_i \ge t \\
      0 & otherwise 
   \end{cases}
\end{equation}

\begin{center}
$u^k_i$ $\Rightarrow$ position of $v_i$ in $p_k$
\end{center}

In (\ref{eq:var_x}) the decision variable $ x^k_{ij} $ is defined to be equal 1 if edge $ e_{ij} $ is traversed by vehicle $ h_k $, and equal 0, otherwise. In (\ref{eq:var_y}) the decision variable $ y^k_{i,t} $ is defined to be equal 1 if at the time $ t $ vehicle $ h_k $ is performing delivery at $v_i $, and equal 0, otherwise. Notice that in (\ref{eq:var_y}) we also introduce $ t^k_i $, the arrival time at node $ v_i $, in order to deduce the stationarity of vehicle $ h_k $ at the time $ t $. Here, $ t^k_i $ can be easily computed by summing travel times and delivery durations that preceded the visit to $ v_i $ in path $ p_k $. Hence, we formalize $ t^k_i $ as follows:

\begin{equation}
\label{eq:arrival_time}
t^k_i = t_k^s + \sum_{v_m \in S_k^i} \sum_{v_n \in S_k^i} (c_{mn} + d_m)x_{ij}
\end{equation}

where $ S_k^i \subseteq D_k $ represents the set of nodes visited in path $ p_k $ before the node $ v_i $, i.e. $ S_k^i = \{ v_j \in D_k : u_j^k \le u_i^k \} $.

Finally, using the introduced notation we formulate our problem as follows:

Minimize
\begin{equation}
\label{eq:objective}
\sum_{p_k \in P}\sum_{v_i\in D_k}\sum_{v_j \in D_k}x^k_{ij}.c_{ij}
\end{equation}

subject to

%Subtour contraints guarantee that there won't be any cycle without the initial node. Because each node has to have an ingoing and an outgoing edge from other edges, there must be at least one cycle. There is no need for additional constraint for the initial point. 

% I agree, at the first look this makes sense...

\begin{equation}
\label{eq:constr1}
\sum_{v_i\in D_k}x^k_{ij}=1, \; \;  \forall v_j \in D_k, \forall p_k \in P
\end{equation}

\begin{equation}
\label{eq:constr2}
\sum_{v_j\in D_k}x^k_{ij}=1, \; \;  \forall v_i \in D_k, \forall p_k \in P
\end{equation}

\begin{equation}
\label{eq:constr3}
\sum_{p_k \in P} y^k_{i,t} \le l_i, \; \; \forall v_i \in \bigcup\limits_{j=1}^{M} D_j, \forall t \in [0,\infty \rangle
\end{equation}

\begin{equation}
\label{eq:constr4}
{2\leq u^k_i \leq |D_k|\;\; \forall i= 2,3,...,|D_k|}
\end{equation}

\begin{equation}
\label{eq:constr5}
\begin{aligned}
 u^k_i - u^k_j +1 \leq (|D_k|-1)(1-x^k_{ij})\\ 
 \forall i= 2,3,...,|D_k|\; \wedge i\neq j \;\;\;\;\\
\end{aligned}
\end{equation}

Objective equation (\ref{eq:objective}) represents the total travel cost over the entire set of solution paths $ P $. Equations (\ref{eq:constr1}) and (\ref{eq:constr2}) ensure that all of the planned nodes for each vehicle are visited exactly once in the corresponding solution paths. Equation (\ref{eq:constr3}) ensures that the number of vehicles at any node does not exceed its capacity at any time. Finally, equations (\ref{eq:constr4}) and (\ref{eq:constr5}) represent Miller-Tucker-Zemlin (MTZ) subtour elimination constraints.

\subsection{Problem complexity}

%{\color{red} I didn't review this part...}

%Traveling salesman problem(TSP) is a benchmark routing problem in computing literature. It is an NP-Hard combinatorial optimization problem trying to minimize the overall travel cost while visiting all given points. Our problem is a multiple TSP problem variation with different initial points and the capacities on delivery spots. In a specific case of our problem, we can have only one vehicle and set of delivery spots. The case exactly refers to TSP, therefore we can deduct our problem is also NP-Hard.

%differences from vehicle routing:
%there is more than one start point of the vehicles
%vehicles visit set may not be discrete as in vehicle routing
%(I don t know where to put these things, after I have read about VRP, I took the notes)
%Vehicle routing problem(VRP) is a well-known combinatorial optimization problem which is NP-Hard. In the problem, basically, there is a depot including all items to be delivered and also a fleet of vehicles to deliver them customers. The aim is to find routes for each of the vehicles by minimizing the total travel cost. In our case, the vehicles are not in the same location at first and the delivery spot set of each vehicle may not be discrete as in VRP. The easier thing in our problem, is that each vehicle is already assigned with set of delivery spots. Its harder part is the possibility of visiting the same spots which have limited capacities.

%However have capacities on spots, no need to assign sets.
%need to mention mTSP but do not understand the difference. it is said it is the relaxed version but how.
%In a specific case of our problem is the basic VRP, where all vehicles are in the same location and have discrete delivery spot set. As it is NP-Hard, we can deduct that our problem is also NP-Hard.%we may explain it better!! noo, it cannot be reduced as in the explanation, bcause vrp also includes choosing which vehicle deliver which item.

% references section

% can use a bibliography generated by BibTeX as a .bbl file
% BibTeX documentation can be easily obtained at:
% http://mirror.ctan.org/biblio/bibtex/contrib/doc/
% The IEEEtran BibTeX style support page is at:
% http://www.michaelshell.org/tex/ieeetran/bibtex/
\bibliographystyle{IEEEtran}
% argument is your BibTeX string definitions and bibliography database(s)
%\bibliography{IEEEabrv,../bib/paper}
%
% <OR> manually copy in the resultant .bbl file
% set second argument of \begin to the number of references
% (used to reserve space for the reference number labels box)
%\bibliography{references}{}

% that's all folks
\end{document}


